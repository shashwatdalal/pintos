\documentclass[a4paper,12pt]{article}
\usepackage [utf8]{inputenc}
\usepackage{xspace} %better spacing option with :
\usepackage{amsmath,amsthm,amssymb} %package matematici
\usepackage{amsfonts} % altri simbolo matematici stronzi
\usepackage{latexsym} %altri simboli chissa quali
\usepackage{graphicx}
\usepackage{listings}
\usepackage{fancyhdr}
\usepackage{multicol}
\usepackage[pdftex]{color}
\usepackage{url}
\usepackage{subfigure}
\usepackage[final]{pdfpages}
\newcommand{\fncyblank }{\fancyhf {}}

\addtolength{\hoffset}{-1,5cm}
\addtolength{\textwidth}{3cm}

\definecolor{Gray}{cmyk}{0,0,0,0.50}

\usepackage{sectsty}
\sectionfont{\large}
\subsectionfont{\normalsize}

\begin{document}

\small

\begin{center}
\begin{LARGE}
Pintos Task 0
\end{LARGE}
\end{center}

\begin{center}
211 Operating Systems \\
Department of Computing \\
Imperial College
\end{center}

This task is logically divided into two parts:
\begin{itemize}
\item Part A: Questions 1-10 test your understanding of Pintos basic concepts. 
      To answer these questions you are invited to carefully read the Pintos manual. 
      For some questions, examining the Pintos code-base might be useful. 
      The maximum possible mark in Part A is 20.
      
\item Part B: tests your comprehension and ability of writing Pintos code. 
      To successfully accomplish this part of the task, you first have to download, compile, install and run Pintos, and then develop a simple functionality. 
      The maximum possible mark in Part B is 30. 
\end{itemize}

\section*{Part A - Codebase Preview (20 marks)}
In this part you are required to answer 10 short questions that test your understanding of the provided Pintos code-base.

\subsubsection*{Question 1 - (1 mark)}
Which Git command should you run to retrieve a copy of your individual repository for Pintos Task 0 in your local directory?\\
(\textit{Hint: be specific to this task.})

\subsubsection*{Question 2 - (1 mark)}
Why is using the {\tt strcpy()} function to copy strings usually a bad idea? \\
(\textit{Hint: be sure to clearly identify the problem.})

\subsubsection*{Question 3 - (1 mark)}
If test \texttt{src/tests/devices/alarm-multiple} fails, where would you find its output and result logs? 
Provide both paths and file names.\\
(\textit{Hint: you might want to run this test and find out.}) 

\subsubsection*{Question 4 - (2 marks)}
In Pintos, a thread is characterized by a struct and an execution stack. 
What are the limitations on the size of these data structures? 
Explain how this relates to stack overflow and how Pintos identifies if a stack overflow has occurred.

\subsubsection*{Question 5 - (6 marks)}
Explain how thread scheduling in Pintos currently works in roughly 300 words. 
Include the chain of execution of function calls. \\
(\textit{Hint: we expect you to at least mention which functions participate in a context switch, how they interact, how and when the thread state is modified and the role of interrupts.)}

\subsubsection*{Question 6 - (1 mark)}
In Pintos, what is the default length (in ticks \emph{and} in seconds) of a scheduler time slice? \\
(\textit{Hint: read the task 0 documentation carefully.})

\subsubsection*{Question 7 - (2 marks)}
In Pintos, how would you print an unsigned 64 bit \texttt{int}? 
(Consider that you are working with C99). 
Don't forget to state any inclusions needed by your code.

\subsubsection*{Question 8 - (2 marks)}
Explain the property of {\bf reproducibility} and how the lack of reproducibility will affect debugging.

\subsubsection*{Question 9 - (2 marks)}
In Pintos, locks are implemented on top of semaphores.
Describe how the functions of a lock are related to those of a semaphore.
What extra property do locks have that semaphores do not?

\subsubsection*{Question 10 - (2 mark)}
Define what is meant by a {\bf race-condition}. Why is the test \texttt{if(x != null)} 
insufficient to prevent a segmentation fault from occurring on an attempted access to a structure through the pointer \texttt{x}?\\
(\textit{Hint: you should assume that the pointer variable is correctly typed, that the structure was successfully initialised earlier in the program 
and that there are other threads running in parallel.})

\newpage
\section*{Part B - The Alarm Clock (30 marks)}

In this part, you are required to implement a simple functionality in Pintos and to answer the design document questions listed below.

\subsection*{Coding the Alarm Clock in Pintos} 
Reimplement \texttt{timer$\_$sleep()}, defined in '\texttt{devices/timer.c}’.\\ 

\noindent Although a working implementation of \texttt{timer$\_$sleep()} is provided, it “busy waits,” that is, 
it spins in a loop checking the current time and calling \texttt{thread$\_$yield()} until enough time has gone by. 
You need to reimplement it to avoid busy waiting. 
Further instructions and hints can be found in the Pintos manual.\\

\noindent The marks for this question are awarded as follows:

Passing the automated tests ({\bf 10 marks}). 

Performance in the Code Review ({\bf 10 marks}). 

Answering the design questions below ({\bf 10 marks}).

\subsection*{Task 0 Design Questions}

\subsubsection*{Data Structures}
A1: ({\bf 2 marks}) Copy here the declaration of each new or changed `\texttt{struct}' or `\texttt{struct}' member, global or static variable, `\texttt{typedef}', or enumeration. Identify the purpose of each in 25 words or less.

\subsubsection*{Algorithms}
A2: ({\bf 2 marks}) Briefly describe what happens in a call to \texttt{timer$\_$sleep()}, including the actions performed by the timer interrupt handler on each timer tick. \\

\noindent A3: ({\bf 2 marks}) What steps are taken to minimize the amount of time spent in the timer interrupt handler?

\subsubsection*{Synchronization}
A4: ({\bf 1 mark}) How are race conditions avoided when multiple threads call \texttt{timer$\_$sleep()} simultaneously? \\ 

\noindent A5: ({\bf 1 mark}) How are race conditions avoided when a timer interrupt occurs during a call to \texttt{timer$\_$sleep()}?

\subsubsection*{Rationale}
A6: ({\bf 2 marks}) Why did you choose this design?  
In what ways is it superior to another design you considered?

\end{document}